\documentclass[11pt]{article}
\usepackage[utf8]{inputenc}
\usepackage[a4paper, total={170mm,257mm}, left=20mm, top=20mm]{geometry}
\usepackage{graphicx}
\usepackage{amsmath}
\usepackage{float}
\usepackage[dvipsnames]{xcolor}
\usepackage{varwidth}
\usepackage{hyperref}

\title{Food delivery from restaurants}
\author{Vitaly Kozhemyak}
\date{January 2022}

\begin{document}
\maketitle
\section{Business research}
\subsection{A top-down estimation of market size}
In this section we consider Moscow as a city. We are going to use a \textbf{TAM-SAM-SOM} methodology to perform an estimation of market size.
\paragraph{TAM.}
According to \href{https://acdn.tinkoff.ru/static/documents/market-for-delivery-food-groceries-and-ready-made-rations.pdf}{\textcolor{blue}{this source}}, for a total of 135B RUB spent last year in Russia. 
%The total number of people who live within the city limits is 12.4 million. In average you spend 1000 RUB on food delivery from resaurants. Suppose that a person makes 3 orders per month.
$$
\text{TAM} = 135 \text{B} \cdot 0.45 \sim 60 \text{B RUB.}
$$
\paragraph{SAM.}
From all people in the city only 65\% are ready to order a food delivery from restaurants
$$
\text{SAM} = \text{TAM} \cdot 0.65 \cdot 0.95 \sim 37 \text{B RUB.}
$$
\paragraph{SOM.}
Top 5 delivery companies
$$
\text{SOM} = \text{SAM} / 5 \sim 7.4 \text{B RUB.}
$$
\subsection{Unit economics with profitability per order}

\section{Data Anaysis}

\subsection{Modeling results}
To compare quailty of different models RMSE metric is used. We perform train-test splitting and use the test dataset to obtain the following results.
\begin{table}[H]
\centering
\begin{tabular}{|p{4cm}|c|c|} 
\hline
\bfseries Model & \bfseries RMSE & \bfseries $\text{R}^2$  \\
\hline\hline
Ridge regression & \textcolor{red}{0.43178} & 0.98734 \\
\hline
LightGBM & \textcolor{Green}{0.40342} & 0.98895 \\
\hline
Random Forest & \textcolor{red}{0.44910} & 0.98631 \\
\hline
Dense neural network & \textcolor{red}{0.55436} & 0.97914 \\
\hline\hline
\bfseries Baseline & 0.41872 & \\
\hline
\end{tabular}
\caption{Result table}
\end{table}
\subsection{Top opportunity to improve the upfront pricing precision}
In addition to main features:
\begin{figure}[H]
\centering
\begin{varwidth}{\linewidth}
\begin{verbatim}
['gps_confidence', 'predicted_distance', 
 'predicted_duration', 'eu_indicator', 
 'overpaid_ride_ticket', 'dest_change_number',
 'prediction_price_type', 'change_reason_pricing',
 'entered_by']
\end{verbatim}
\end{varwidth}
\caption{Main features}
\end{figure}
I suggest we use features such as
\begin{figure}[H]
\centering
\begin{varwidth}{\linewidth}
\begin{verbatim}
['day_of_week', 'is_weekend', 
 'part_of_day', 'speed', 
 'isairport_2000', 'isairport_6000']
\end{verbatim}
\end{varwidth}
\caption{Additional features}
\end{figure}
As I mentioned before in \textit{Data\_Analysis.ipynb} that the next features
\texttt{['isairport\_2000', 'isairport\_6000']} could be derived from GPS-coordinates. Also if I knew pickup and dropoff coordinates I would add:
\begin{itemize}
\item weather conditions,
\item the most visited places,
\item good/bad neighbourhoods,
\item some data from similar ride-hailing applications.
\end{itemize}
\end{document}