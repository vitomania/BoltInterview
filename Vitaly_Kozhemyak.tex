\documentclass[11pt]{article}
\usepackage[utf8]{inputenc}
\usepackage[a4paper, total={6in, 8in}]{geometry}
\usepackage{graphicx}
\usepackage{amsmath}
\usepackage{float}
\usepackage{xcolor}
\usepackage{hyperref}

\title{Food delivery from restaurants}
\author{Vitaly Kozhemyak}
\date{January 2022}

\begin{document}
\maketitle
\section{A top-down estimation of market size}
In this section we consider Moscow as a city. We are going to use a \textbf{TAM-SAM-SOM} methodology to perform an estimation of market size.
\subsection{TAM}
According to \href{https://acdn.tinkoff.ru/static/documents/market-for-delivery-food-groceries-and-ready-made-rations.pdf}{\textcolor{blue}{this source}}, for a total of 135B RUB spent last year in Russia. 
%The total number of people who live within the city limits is 12.4 million. In average you spend 1000 RUB on food delivery from resaurants. Suppose that a person makes 3 orders per month.
$$
\text{TAM} = 135 \text{B} \cdot 0.45 \sim 60 \text{B RUB.}
$$
\subsection{SAM}
From all people in the city only 65\% are ready to order a food delivery from restaurants
$$
\text{SAM} = \text{TAM} \cdot 0.65 \cdot 0.95 \sim 37 \text{B RUB.}
$$
\subsection{SOM}
Top 5 delivery companies
$$
\text{SOM} = \text{SAM} / 5 \sim 7.4 \text{B RUB.}
$$
\section{Unit economics with profitability per order}
\begin{table}[H]
\centering
\begin{tabular}{|p{3cm}|p{10cm}|} 
\hline
\bfseries Parameter & \bfseries Description  \\
\hline\hline
DATE & Given in YYYY-mm-dd format.\\ 
\hline
avg temp & Average temperature for the day in degrees Celsius to tenths.\\
\hline
max temp & Maximum temperature reported during the day in Celsius to tenths.\\
\hline
min temp & Minimum temperature reported during the day in Celsius to tenths.\\
\hline
dew point & Average dew point for the day in degrees Celsius to tenths.\\
\hline
precipitation & Total precipitation (rain and/or melted snow) reported during the day in millimeters and hundredths.\\
\hline
max wind speed & Maximum sustained wind speed reported for the day in kmh to tenths.\\
\hline
condition & Occurrence during the day of: Fog, Rain or Drizzle, Snow or Ice Pellets, Hail, Thunder, Tornado or Funnel Cloud.\\
\hline
\end{tabular}
\caption{Parameter descriptions.}
\label{table:1}
\end{table}
\section{Units of measure}
\begin{table}[H]
\centering
\begin{tabular}{|c|c|} 
\hline
\bfseries Parameter & \bfseries Metric units  \\
\hline\hline
Temperature & c (celsius)\\ 
\hline
Wind Speed & km (kilometer/hour)\\ 
\hline
Precipitation Amount & mm (millimeters)\\ 
\hline
\end{tabular}
\caption{Units of measure.}
\label{table:1}
\end{table}
\section{API information}
\url{https://www.ncei.noaa.gov/support/access-data-service-api-user-documentation}
\section{Additional information}
\begin{itemize}
	\item Main source: \url{https://www.ncei.noaa.gov}
	\item Description of the raw data: \url{https://www.ncei.noaa.gov/data/global-summary-of-the-day/doc/readme.txt}
	\item Airport information: \url{https://www.ncei.noaa.gov/pub/data/noaa/isd-history.txt}
	\item Country abbreviations: \url{https://www.ncei.noaa.gov/data/global-summary-of-the-day/doc/country-list.txt}
\end{itemize}
\section{Examples}
\begin{figure}[H]
\centering
\includegraphics[width=\textwidth]{images/TEMP.png}
\caption{MAX, AVG, MIN temperature. Units of measurements (Celsius)}
\end{figure}
\begin{figure}[H]
\centering
\includegraphics[width=\textwidth]{images/MXSPD.png}
\caption{Max wind speed (kmh)}
\end{figure}
\begin{figure}[H]
\centering
\includegraphics[width=\textwidth]{images/PRCP.png}
\caption{Precipitaion (mm)}
\end{figure}
\end{document}